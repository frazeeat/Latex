\documentclass{article}
\usepackage[margin=0.5in]{geometry}
\begin{document}
	\begin{flushleft}
		Adam Frazee \\*
		CSE 278\\*
		Dr.Yue\\*
		11/20/2015\\*
	\end{flushleft}
	\begin{center}
		\LARGE{\textbf{Homework 7}}
	\end{center}
	\textbf{4.3} When processor designers consider a possible improvement to the processor
datapath, the decision usually depends on the cost/performance trade-off . In
the following three problems, assume that we are starting with a datapath from
Figure 4.2, where I-Mem, Add, Mux, ALU, Regs, D-Mem, and Control blocks have
latencies of 400 ps, 100 ps, 30 ps, 120 ps, 200 ps, 350 ps, and 100 ps, respectively,
and costs of 1000, 30, 10, 100, 200, 2000, and 500, respectively.
Consider the addition of a multiplier to the ALU. Th is addition will add 300 ps to the
latency of the ALU and will add a cost of 600 to the ALU. The result will be 5\% fewer
instructions executed since we will no longer need to emulate the MUL instruction.
\\*
\begin{center}
  \begin{tabular}{ |l | l | l | l | l | l | l  | l || l | }
    \hline
    &I-Mem & Add & Mux & ALU & Regs & D-Mem & Control Blocks & Sum \\ \hline
    Time&400ps & 100ps & 30ps & 120ps & 200ps &350ps & 100ps& 1300ps\\ \hline
    Cost&1000 & 30*2 & 10*3 & 100 & 200 & 2000 & 500 & 3890 \\
    \hline
  \end{tabular}
\end{center}

\textbf{4.3.1} What is the clock cycle time with and without this improvement?
\begin{center}



I-Mem + Add + Mux + ALU + Regs + D-Mem + Control blocks \\
400 + 100 + 30 + 120 +200 +350 + 100 = 1300ps

\end{center}

\textbf{4.3.2} What is the speedup achieved by adding this improvement?

\textbf{4.3.3}Compare the cost/performance ratio with and without this
improvement.
\end{document}