\documentclass[12pt]{article}
\usepackage[margin=1.0in]{geometry}
\usepackage{setspace}

\bibliographystyle{unsrtnat}
\begin{document}
	\begin{flushleft}
		Adam Frazee \\*
		BIO 131\\*
		Cheryl Smyser\\*
		02/17/2016\\*
	\end{flushleft}

\begin{center}
 \Large{Fracking}
\end{center}
\doublespacing

Hydraulic fracturing, or fracking is a process in which to pump gas, or oil out of shale rocks. Fracking has pros and cons associated with it. Some pros are covered in the article written by Hassett and Mathur published authors in the \textit{Oxford Energy Forum}, where they go over the benefits that fracking has done to our economy. Which was published in February of 2013. On the other side we have Hofman a geologist from Montana State University talking about how fracking can negatively effect the environment. Which was published in 2012.

First let's talk about the positives of fracking. Which is discussed in Hassett and Mathur's article. They look at the economic virtues of fracking, and some of the ecological benefits of fracking. In an attempt to show fracking in as best of light as possible. 

On the economics side of things there is a direct impact on our economy, and an indirect benefit on our world . As a direct impact it creates jobs "This increase in value produced can also increase the number of people employed directly in production and delivery activities." (Hassett). This point is then supported by showing how the in the 2008 recession that the oil industry was still employing people "As
other industries have sputtered in the
aftermath of the 2008 recession, oil and gas has been a remarkably bright spot in the US economy, with employment at the end of 2012 at its highest since 1987"(Hassett).

Indirectly it can effect the world we live in. With Electric generation being mostly coal based. Gasses extracted from fracking, burning cleaner than coal. "According to the Environmental Protection Agency, natural gas-fired electricity generates half the carbon dioxide of coal-fired production. An estimate of the indirect benefit of fracking should include an estimate of the potential social gains from this reduction." (Hassett). 
 
On the other side of the shale, we have a geologist point of view. Which contrasts the view of Hasset and Mathur. Hoffman focuses on the geological effects of fracking. Some of the main points that Hoffman brings up are water, soil and air pollution caused by fracking. Also how fracking could affect the human body.

Natural gasses main component is 
methane gas. It is also a worse gas than $CO_2$ as a green house gas. It is estimated that 4\% of the methane produced by fracking goes into the atmosphere (Hoffman). A main component of fracking is water. Additives are mixed in, and the exact recipe is often a trade secret. Some of the things in the water are "containing not only the added chemicals, but other naturally occurring radioactive material, liquid hydrocarbons, brine water and heavy metals."(Hoffman). It is said that 75\% of the chemicals can "affect the skin, eyes,and other sensory organs, and the respiratory and gastrointestinal systems." (Hoffman). Finally the soil, when affected by oil spills becomes waste, and unusable. These spills are becoming more common, and sometimes unreported(Hoffman).

Both articles bring up good points on their respective fronts. Hassett and Mathur claims that fracking is to an economic benefit is true. However their idea that fracking is good for the environment is based off of weak facts. The article that Hoffman wrote was a scholarly article which had many references to other articles. Which is a good basis for an article, and a good enough reason to support their claim.  

\singlespacing
\noindent \large{Sources}\\
KEVIN HASSETT and APAMA MATHUR "Benefits of Hydraulic Fracking" Electronic 
2/17/2016 www.aei.org/wp-content/uploads/2013/04/-benefits-of-hydraulic-fracking\_095955248581.pdf 
\\
\\
Joe Hoffman "Potential Health and Environmental Effects of Hydrofracking in the Williston Basin, Montana" Electronic serc.carleton.edu/ NAGTWorkshops/health/case\_studies/hydrofracking\_w.html
\end{document}