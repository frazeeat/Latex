\documentclass[12pt]{article}
\usepackage[margin=1.0in]{geometry}
\usepackage[square,numbers]{natbib}

\usepackage{url}
\usepackage{setspace}

\bibliographystyle{unsrtnat}
\begin{document}
	\begin{flushleft}
		Adam Frazee \\*
		BIO 131\\*
		Cheryl Smyser\\*
		02/03/2016\\*
	\end{flushleft}

\begin{center}
 \Large{Bats are Up}
\end{center}
\doublespacing

The Indiana bat is dying off. In fact a ton of bats are dying off, due to a condition called the white-nose syndrome. This syndrome is caused by a fungus that lives in the soil, and can be tracked into a bats habitat. The main culprit in habitat contamination are humans. Bats are being studied to see if they can carry deadly diseases. While watching species disappear is sad, if that species can do a ton of harm to a human population, should they be left to die? 

Species going extinct is a way of life on this world. The one that is burned into the mind of every child is dinosaurs. It is impossible to say that the world would be a better place, with or without dinosaurs. In the same way it is impossible to say that the world would be better or worse off without any species. The world is a complex mesh of plants, and animals. Taking out a thread, can lead to varying and unpredictable results, which can't be covered in this article. However what is different with bats compared with dinosaurs, is we are given a choice. We have the ability to stop contaminating the bats habitat, something the event that wiped out the dinosaurs never thought of. What is happening with the bats?

Humans have a desire for exploration "Curious hikers and weekend explorers crawl around in those caves or in any of the hundreds of abandoned mines throughout the state, unknowingly picking up white-nose syndrome spores on their clothing and spreading the disease." \cite{dispatch}. We aren't actively trying to kill off bats, but we are a reason that they are dying off. Bats also have a very low reproduction rate \cite{dispatch}. So it will be hard for the bats to rebound even if a cure was found.

Things are being done to help the bats. Things like this "Consequently, the U.S. Forest Service extended an order last week prohibiting anyone from entering underground mines or tunnels in Wayne National Forest, which covers more than 800,000 acres in three regions in Ohio encompassing Ironton, Athens and Marietta." \cite{dispatch}. Many caves have been closed that harbored bats, and overall we have made the decisions to save the bats.

What if bats are a serious threat to humans? That is a serious concern, and still being studied. Some of the desises that can find refuge in bats are the "Ebola, rabies, Marburg and the SARS coronavirus"\cite{wired}. A nice parallel is with the common mosquito. Malaria has caused many deaths, and mosquito's where a major factor.  Another example is with rats "Bats and other species that chronically harbor viruses, such as rats or mice, are known as disease reservoirs. Most of the time, these reservoirs stay intact, with infected animals rarely showing symptoms of disease. But sometimes they leak, letting a virus infect new, much more vulnerable species."\cite{wired}. The research is still ongoing to know if bats are actually a risk factor for carrying extremely deadly diseases.

Humans are destroying bat populations through contaminating their habitat. While this is something to be aware of, whether to protect bats is a matter of opinion. As humans we should protect and preserve the world around us. Should we do that, if it means risking the lives of other humans? The research is still not clear about the true risk of bats, and carrying deadly viruses. It however should be noted, and thought about so as a species ourselves we can come up with a solution.

\bibliographystyle{abbr}
\bibliography{bibTex}
\end{document}