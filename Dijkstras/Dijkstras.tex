\documentclass{article}
\usepackage[margin=0.5in]{geometry}
\usepackage{color}
\usepackage{listings}
\lstset{ %
language=C++,                % choose the language of the code
basicstyle=\footnotesize,       % the size of the fonts that are used for the code
numbers=left,                   % where to put the line-numbers
numberstyle=\footnotesize,      % the size of the fonts that are used for the line-numbers
stepnumber=1,                   % the step between two line-numbers. If it is 1 each line will be numbered
numbersep=5pt,                  % how far the line-numbers are from the code
backgroundcolor=\color{white},  % choose the background color. You must add \usepackage{color}
showspaces=false,               % show spaces adding particular underscores
showstringspaces=false,         % underline spaces within strings
showtabs=false,                 % show tabs within strings adding particular underscores
frame=single,           % adds a frame around the code
tabsize=2,          % sets default tabsize to 2 spaces
captionpos=b,           % sets the caption-position to bottom
breaklines=true,        % sets automatic line breaking
breakatwhitespace=false,    % sets if automatic breaks should only happen at whitespace
escapeinside={\%*}{*)}          % if you want to add a comment within your code
}
\usepackage{amsmath}
\usepackage{algorithm}
\usepackage[noend]{algpseudocode}
\begin{document}
	\begin{flushleft}
		Adam Frazee \\*
		CSE 464\\*
		Gregory Gelfond\\*
		11/13/2015\\*
	\end{flushleft}
	\begin{center}
		\LARGE{\textbf{Dijkstra's Algorithm}}
	\end{center}
	\textbf{Formulate The Problem:} Dijkstra's Algorithm is an algorithm used to determine the shortest path between two nodes. In this algorithm we have a graph (G) which is a data structure for holding nodes or (V) vertices, and also has Edges (E). E's have a weight or cost associated with them (W). The source and destination are both nodes in the graph G. They are connected through a path of E's.Another way of looking at Dijkstra's Algorithm is as finding a path of V's where the sum E's W associated with those V's is the minimum possible.  \\ \\*
		\large{\textbf{Pseudo Code:}
	\begin{algorithm}
		\caption{Dijkstra}
		\begin{algorithmic}[1]
			\Procedure{Dijkstra}{G,$V_{Source}$,$V_{Destination}$}
		
			
			\EndProcedure
		\end{algorithmic}
	\end{algorithm}
\end{document}